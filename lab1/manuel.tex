\documentclass{article}

\begin{document}
\textbf{\textit{Introduction}}

For the first project, it was proposed the creation and implementation of a grammar, which is supposed to take as input a LaTex file, usually used to create papers and scientific articles, and then, convert those files to HTML format as output, that is used to creation and display of web pages.

To do this, it was created a LALR(1) grammar, taking a finite automata that uses a stack to parse. The implementation was done using Flex and Bison tools (an lexical analyser and parser respectively). Not all of the LaTeX commands had effect in the HTML scenario, so, some of then should be just read and others interpreted.

\textit{\textbf{Supported Commands}}

The list of implemented commands, with all the respective results in HTML:

\begin{itemzie}

	\item \textbf{a) Header Commands}
		\begin{itemize}
			 \item documentclass[...]{documentname}: recognized, but not effective
			 \item usepackage[...]{nomepacote}:  recognized, but not effective
			 \item title{titulo}: define the title of the LaTex document, so it could be displayed with \textbf{maketitle} command;
			 \item author{nomeautor}: não exerce efeito sobre o código HTML, apesar de ser reconhecido;
		\end{itemize}

	\item \textbf{b) Body Commands:}
		\begin{itemize}
	 		\item \textit{begin{document}:} marks the begining of the document;
	 		\item \textit{end{document}:} marks the ending of the document;
	 		\item \textit{maketitle: mostra o} shows the title defined in \"title\" command in the header;
	 		\item \textit{textbf{texto1}:} apply bold to texto1;
	 		\item \textit{textit{texto1}:} aply italic to texto1;
	 		\item \textit{begin{itemize}:} generate an enumerated list of itens (treat nested list);
	 		\item \textit{item texto1:} includes texto1 as an item inside the itens list; 
	 		\item \textit{end{itemize}:} marks the end of the item list;
	 		\item \textit{includegraphics{figura1}:} mostrar a imagem figura1;
	 		\item \textit{cite{ref1}:} adiciona o número da referência ref1, que se encontra na seção de \'thebibliography\';
	 		\item \textit{begin{thebibliography}:} inicia uma seção de bibliografias com uma lista de referências;
	 		\item \textit{bibitem{ref1}:} define um item de referência;
	 		\item \textit{end{thebibliography}:} adiciona as referências bibliográficas (o corpo do documento poderá conter apenas uma seção \'thebibliography\');
	 		\item \textit{\$\$: o texto entre} dois caracteres dólar serve para fazer a marcação de um ambiente matemático;
		\end{itemize}

\end{itemize}

\textit{\textbf{Gramática Desenvolvida}}

Em LaTex, apesar de haver uma certa ordem com a qual aluguns comandos estão sujeitos, em geral o texto que deve ser impresso no documento final está misturado com comandos que servem para formatar esse texto (como por exemplo os comandos de texto em negrito e itálico). Buscou-se fazer uma gramática que fosse simples o suficiente para não gerar conflitos e ao mesmo tempo tivesse a capacidade de analisar comandos e textos misturados, bem como modos de formatação (texto normal ou matemático, através do marcador \$).

[FOTO DA GRAMÁTICA SEGUE AQUI]

Para lídar com o modo matemático, contou-se com o auxílio da biblioteca MathJax. Através um processamento de texto HTML de duas fases, a biblioteca realiza na primeira fase uma analise do conteúdo do arquivo HTML em busca de marcadores de ambiente matemático, e após isso, realiza a devida substituição do texto não formatado em ambiente matemático (demarcado por dois caracteres \'\$\' no caso do projeto) para fórmulas.

Comandos Para a Execução
\end{document}
