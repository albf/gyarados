\documentclass{article}

\begin{document}
\textbf{\textit{Introdução}}

Para o projeto 1, foi proposta a criação de uma gramática, bem como sua implementação, a qual receberia como entrada um arquivo no formato de LaTeX, comumente utilizada para criação de papers e artigos científicos, e o converteria para um arquivo simples no formato HTML, utilizado principalmente para criação de páginas web.
Para isso, foi criada uma gramática do tipo LALR(1), que utiliza um autômato de pilha para fazer o parsing. A implementação da gramática foi feita com o uso das ferramentas Flex e Bison, um analisador léxico e um parser respectivamente, popularmente utilizados para este fim. O desafio era que alguns comandos básicos do LaTex deveriam ser interpretados, apesar de nem todos terem efeito de fato na saída, de maneira a suportar certas funcionalidades LaTex para o HTML.

Comandos Suportados

A lista de comandos que foram implementados, bem como os seus respectivos resultados no HTML são os seguintes:

\begin{itemzie}

	\item \textbf{a) Comandos de cabeçalho:}
		\begin{itemize}
			 \item documentclass[...]{nomedocumento}: não exerce efeito sobre o HTML, apesar de ser reconhecido;
			 \item usepackage[...]{nomepacote}:  não exerce efeito sobre o código HTML, apesar de ser reconhecido;
			 \item title{titulo}: guarda o Título do documento LaTex, para que seja mostrado no HTML caso o comando \textbf{maketitle} seja dado;
			 \item author{nomeautor}: não exerce efeito sobre o código HTML, apesar de ser reconhecido;
		\end{itemize}

	\item \textbf{b) Comandos de documento:}
		\begin{itemize}
	 		\item \textit{begin{document}:} demarca o início do corpo do documento LaTex;
	 		\item \textit{end{document}:} demarca o final do documento LaTex;
	 		\item \textit{maketitle: mostra o} título descrito pelo comando \"title\" no cabeçalho (supõe-se que para que sempre que o comando é dado, existe um título definido para o documento na seção de cabeçalho);
	 		\item \textit{textbf{texto1}:} aplica negrito em texto1;
	 		\item \textit{textit{texto1}:} aplica itálico em texto1;
	 		\item \textit{begin{itemize}:} demarca e gera uma lista não enumerada com os itens (pode haver encadeamento de lista);
	 		\item \textit{item texto1: inclui} texto1 como um item dentro de uma lista de itens; 
	 		\item \textit{end{itemize}:} demarca o final da lista de itens;
	 		\item \textit{includegraphics{figura1}:} mostrar a imagem figura1;
	 		\item \textit{cite{ref1}:} adiciona o número da referência ref1, que se encontra na seção de \'thebibliography\';
	 		\item \textit{begin{thebibliography}:} inicia uma seção de bibliografias com uma lista de referências;
	 		\item \textit{bibitem{ref1}:} define um item de referência;
	 		\item \textit{end{thebibliography}:} adiciona as referências bibliográficas (o corpo do documento poderá conter apenas uma seção \'thebibliography\');
	 		\item \textit{\$\$: o texto entre} dois caracteres dólar serve para fazer a marcação de um ambiente matemático;
		\end{itemize}

\end{itemize}

\textit{\textbf{Gramática Desenvolvida}}

Em LaTex, apesar de haver uma certa ordem com a qual aluguns comandos estão sujeitos, em geral o texto que deve ser impresso no documento final está misturado com comandos que servem para formatar esse texto (como por exemplo os comandos de texto em negrito e itálico). Buscou-se fazer uma gramática que fosse simples o suficiente para não gerar conflitos e ao mesmo tempo tivesse a capacidade de analisar comandos e textos misturados, bem como modos de formatação (texto normal ou matemático, através do marcador \$).

[FOTO DA GRAMÁTICA SEGUE AQUI]

Para lídar com o modo matemático, contou-se com o auxílio da biblioteca MathJax. Através um processamento de texto HTML de duas fases, a biblioteca realiza na primeira fase uma analise do conteúdo do arquivo HTML em busca de marcadores de ambiente matemático, e após isso, realiza a devida substituição do texto não formatado em ambiente matemático (demarcado por dois caracteres \'\$\' no caso do projeto) para fórmulas.

Comandos Para a Execução
\end{document}