\documentclass{article}

\begin{document}
\textbf{\textit{Introduction}}

For the first project, it was proposed the creation and implementation of a grammar, which is supposed to take as input a LaTex file, usually used to create papers and scientific articles, and then, convert those files to HTML format as output, that is used to creation and display of web pages.

To do this, it was created a LALR(1) grammar, taking a finite automata that uses a stack to parse. The implementation was done using Flex and Bison tools (an lexical analyser and parser respectively). Not all of the LaTeX commands had effect in the HTML scenario, so, some of then should be just read and others interpreted.

\textit{\textbf{Supported Commands}}

The list of implemented commands, with all the respective results in HTML:

\begin{itemize}

	\item \textbf{Header Commands}
		\begin{itemize}
			 \item documentclass: recognized, but not effective
			 \item usepackage:  recognized, but not effective
			 \item title: define the title of the LaTex document, so it could be displayed with \textbf{maketitle} command;
			 \item author: recognized, but not effective
		\end{itemize}

	\item \textbf{Body Commands:}
		\begin{itemize}
	 		\item \textit{begin}: marks the begining of the document;
	 		\item \textit{end:} marks the ending of the document;
	 		\item \textit{maketitle: mostra o} shows the title defined in \"title\" command in the header;
	 		\item \textit{textbf:} apply bold to texto1;
	 		\item \textit{textit}: apply italic to texto1;
	 		\item \textit{begin}: generate an enumerated list of itens (treat nested list);
	 		\item \textit{item texto1:} includes texto1 as an item inside the itens list; 
	 		\item \textit{end}: marks the end of the item list;
	 		\item \textit{includegraphics:} shows figura1;
	 		\item \textit{cite:} add the number of the ref that is in the section \'thebibliography\';
	 		\item \textit{begin:} begin the section of bibliography as a list of references;
	 		\item \textit{bibitem:} define a reference item;
	 		\item \textit{end:} add the references to the body of the document (one per document);
	 		\item \textit{DOLLAR:} the text between two dollars serves to mark the math environment;
		\end{itemize}

\end{itemize}

\textit{\textbf{The Grammar}}

In LaTex, despite some commands follow an order, in general the text that must be print in the final document is mixed with the commands that format this same text (for instance, commands like bold os italic). We attempted to do a grammar that is simple enough to do not have conflicts, but powerfull enough, at the same time to analyse commands, text and math mode (using DOLLAR).

To guide the process, we used some principles of construction grammar described in \cite{Appel02} and \cite{Aho86}.

\includegraphics{monarch.jpg}

\begin{thebibliography}

\bibitem{Appel02} Appel, A., \textbf{Modern Compiler Implementation in Java, 2nd Ed.}, October 21, 2002

\bibitem{Aho86} Aho, A., Sethi, M., Ullman, J., \textbf{Compiladores : Princípios, Técnicas e Ferramentas}, 1986

\end{thebibliography}

\textit{\textbf{The Math Text}}
To deal with the math mode, we used MathJax library. Using an two-phase processing of the HTML text, the library check the content looking for makers DOLLAR that define the math environment, and then, it substitute of the text in LaTex format to the correct Math format and formulas.

For example, the following text is a math text:

"When $a \ne 0$, there are two solutions to $ax^2 + bx + c = 0$ and they are $x = {-b \pm \sqrt{b^2-4ac} \over 2a}.$"


\end{document}
